\documentclass[11pt,a4paper]{paper}

\usepackage[margin=2cm]{geometry}
\usepackage[utf8]{inputenc}

\usepackage[sort]{natbib}

\title{Activity prediction via movement sensor}
\author{Bakhodir Ashirmatov, Benjamin Quack and Qin Zhao}

\begin{document}


\maketitle

\section{Related work}

A lot of research has already been conducted in the field of human activity recognition via
 accelerometer sensors and already there are established ways of approaching the corresponding data \cite{bullingetal2014}.
Usually the so called activity recognition chain has been used to recognizes activities. 
It comprises data acquisition, signal processing and segmentation, feature extraction and selection,
 training, and classification.
Many types of activities have already been studied, mostly regarding sport and fitness activities, household activities (e.g. vacuuming),
 everyday activities (e.g. reading a newspaper) or the differentiation between different types 
 of movements (e.g. moving stairs, running, ambulation, laying down) \cite{banosetal2014}.
However, the specific prediction of and differentiation between computer-related activities have not been studied yet to 
 our knowledge.
Only a few studies even include activities like writing on keyboard or typing but only compare those with other non-computer-related 
 activities like the aforementioned ones.
These studies are briefly reviewed.
Also of interested to our project are those studies that assess the possibilities of only using a wristband sensor in comparison 
 to multiple sensors around the body (ankles, legs, upper arms, shoulder and so forth).
 
A study by Kern, Schiele and Schmidt \cite{kernetal2003} includes activities like sitting, standing, walking, writing on a whiteboard and shaking
 hands beside typing on a keyboard.
They apply a Bayesian classifier in order to predict the different activities.
Sensors on the six locations (knee, ankle, hip, wrist, elbow and shoulder) provide the data.
Interestingly, typing on a keyboard can be identified fairly well by utilizing only the wrist sensor.

Mannini and colleagues \cite{manninietal2013} studied four main activity classes:
 sedentary, cycling, ambulation and other activities. 
Sedentary activities also include two types of computer-related activities, typing and 
 internet search.
They use support vector machines for classification. 
Both computer-related activities can be classified fairly well 
 belonging to the class of sedentary activities using the sensor 
 information from the wrist (91.7 \% internet search and 87.9 \% typing).
However, no results on the differentiation between the two types
 has been reported.

Another study by Suutala, Pirttikangas and Röning \cite{suutalaetal2007} also 
 includes typing beside 16 other types of activities. 
They use four sensors (right thigh and wrist, left wrist and necklace).
Results specifically addressing the recognition of typing are not provided.
In contrast to both previous studies, activities were recorded within the
 natural environment of the participant and not instructed in a laboratory setting.
 
TODO \cite{huynhetal2007}


 


 




\bibliographystyle{plain}
\bibliography{bib/references.bib}

\end{document}